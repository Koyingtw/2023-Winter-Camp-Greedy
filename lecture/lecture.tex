\documentclass[aspectratio=169]{beamer}
\usefonttheme{serif}
\usepackage{xeCJK}
\usepackage{fontspec}
\usepackage{graphicx}
\usepackage{listings}
\usepackage{xcolor}
\usepackage{indentfirst}
\usepackage{tikz}
\usepackage{amssymb}
\usepackage{amsthm}
\usepackage{amsmath}
\usepackage{tabularx}
\usepackage{hyperref}
\usepackage{ulem}
\usepackage{version}
\usepackage{thmtools}
\usepackage{qtree}
\usepackage{algpseudocode}
\usepackage{mathtools}
\usepackage{multicol}
\usepackage{xcolor}
\usepackage{ulem}

\AtBeginDocument{%
    \DeclareSymbolFont{pureletters}{T1}{lmr}{\mddefault}{it}%
}

\XeTeXlinebreaklocale "zh"
\XeTeXlinebreakskip = 0pt plus 1pt

\setCJKmainfont{NotoSansTC-Medium.otf}
\setmainfont{JetBrainsMono-SemiBold.ttf}

\usetikzlibrary{arrows,decorations.markings,decorations.pathreplacing}
\newenvironment{Hint}{\noindent\textbf{Hint.}}{}

\tikzstyle {graph node} = [circle, draw, minimum width=1cm]
\tikzset{edge/.style = {decoration={markings,mark=at position 1 with %
            {\arrow[scale=2,>=stealth]{>}}},postaction={decorate}}}

\lstset{
    basicstyle=\ttfamily\normalsize,
    numberstyle=\normalsize,
    numbers=left,
    stepnumber=1,
    numbersep=3pt,
    commentstyle=\color{black!50},
    keywordstyle=\color{white!0!blue},
    stringstyle=\color{black!50!green},
    showspaces=false,
    showstringspaces=false,
    showtabs=false,
    tabsize=4,
    captionpos=b,
    breaklines=true,
    breakatwhitespace=false,
    escapeinside={\%*}{*)},
    morekeywords={*}
}

\AtBeginSection[]{
  \begin{frame}
  \vfill
  \centering
  \begin{beamercolorbox}[sep=8pt,center,shadow=true,rounded=true]{title}
    \usebeamerfont{title}\insertsectionhead\par%
  \end{beamercolorbox}
  \vfill
  \end{frame}
}

\title{貪心 Greedy}
\subtitle{2023 SCIST x NHDK x 南 11 校寒訓}
\author{Koying}
\date{2023-01-30}

\usetheme{Madrid}
\usecolortheme{default}
\setbeamertemplate{itemize items}[square]
\setbeamertemplate{enumerate items}[default]
\setbeamertemplate{blocks}[default]
\lstdefinestyle{myStyle}{
    belowcaptionskip=1\baselineskip,
    breaklines=true,
    frame=none,
    numbers=none, 
    basicstyle=\footnotesize\ttfamily,
    keywordstyle=\bfseries\color{green!40!black},
    commentstyle=\itshape\color{purple!40!black},
    identifierstyle=\color{blue},
    backgroundcolor=\color{gray!10!white},
}

\begin{document}

    \begin{frame}
        \titlepage
    \end{frame}

    \begin{frame}
        \includegraphics[width=\textwidth]{img/SCIST_Sponser.png}
    \end{frame}

    \begin{frame}{協辦單位:ITSA}
        \begin{center}
            \includegraphics[width=0.5\textwidth]{img/ITSA.png}
        \end{center}
    \end{frame}

    \begin{frame}{在開始之前}
        \begin{itemize}
            \item<1-> 把個資攤開來,就不會有個資問題了
            \item<2-> FB:Weak Koying
            \item<2-> IG:cisc.\_.koying
            \item<2-> GitHub:Koyingtw
            \item<2-> \href{https://koyingtw.github.io/}{blog}
            \item<2-> \href{https://github.com/Koyingtw/spec-share}{我的清大備審}
            \item<2-> 特選錄取有報的所有資工系,\href{https://koyingtw.github.io/2022/10/31/特選心得/}{特選心得彙整與各系分數}
            \item<2-> 我的特選心得文有被刊登到\href{https://www.unews.com.tw/News/Info/6066}{大學問網站},想聽更多故事的話,歡迎來第五天的經驗分享(如果只參加前三天,想來後兩天的可以來登記)
        \end{itemize}
    \end{frame}

    \begin{frame}{目錄}
        \begin{itemize}
            \item 貪心入門
            \item 交換貪心
            \item 一些經典問題
        \end{itemize}
    \end{frame}

    \begin{frame}{貪心入門}
        \begin{itemize}
            \item<1-> 什麼是貪心?
            \item<2-> 人的本性?
            \item<3-> 確實
            \item<4-> 有沒有更精確的描述方法?
            \item<5-> 我的答案是:不斷選擇「目前」最好的選項
        \end{itemize}
    \end{frame}

    \begin{frame}{來一個簡單的題目}
        \begin{block}{硬幣問題 1}
            你有無限多個面額為 $1, 5, 10, 50$ 的硬幣,請問要怎麼用最少的硬幣數量求出 $n$ 元?
        \end{block}
    \end{frame}

    \begin{frame}{硬幣問題}
        \begin{itemize}
            \item<1-> 相信有買過東西的人應該都知道該怎麼辦
            \item<1-> 顯然就是先拿面額大的嘛
            \item<2-> 但,這個策略難道永遠都是正確的嗎?
        \end{itemize}
    \end{frame}

    \begin{frame}{硬幣問題}
        \begin{block}{硬幣問題 2}
            你有無限多個面額為 $1, 5, 11$ 的硬幣,請問要怎麼用最少的硬幣數量求出 $n$ 元?
        \end{block}

        \begin{itemize}
            \item<1-> 使用剛剛的策略還會正確嗎?
            \item<2-> 假設 $n = 15$,使用剛剛的策略會拿 $11 + 1 + 1 + 1 + 1$
            \item<2-> 但是 $5 + 5 + 5$ 才是最少的硬幣數量
            \item<3-> 代表著這個策略並不是永遠正確的,或許只有在硬幣面額呈倍數時才成立
            \item<4-> 那麼該如何證明呢?以下會介紹一些在離散數學中比較常見的證明方式
        \end{itemize}
    \end{frame}

    \section{貪心法的常見證明方式}

    \begin{frame}{反證法}
        \begin{itemize}
            \item<1-> proof by contradiction
            \item<1-> 給出某命題 $p$ 與 $\bar{p}$(非 $p$),其滿足排中率(${\displaystyle (p\vee \neg p)}$ 為真,
            也就是 $p$ 與非 $p$ 至少有一為真)
            \item<2-> 假設 $\bar{p}$ 成立,但經過推導後我們發現 $\bar{p}$ 並不成立,那麼就代表 $p$ 成立
        \end{itemize}
    \end{frame}

    \begin{frame}{反證法}
        \begin{block}{$\sqrt{2}$ 為無理數的證明}
            請證明 $\sqrt{2}$ 為無理數(無法使用兩互質整數 $p, q$ 將其表示為 $\frac{p}{q}$)
        \end{block}

        \begin{itemize}
            \item<1-> 這個反證法的例子大家高中應該都有學過
            \item<2-> 假設 $\sqrt{2}$ 為有理數,那麼根據有理數的性質,$\sqrt{2} = \frac{p}{q}$
            \item<3-> 移項之後:$2q^2 = p^2$
            \item<4-> 則 $2 \mid p^2$,又因此可推導出 $2 \mid p$,所以我們可以將 $p$ 寫成 $2k$
            \item<5-> 再套回去剛剛的式子:$q^2 = 2k^2$,因此 $2 \mid q$
            \item<6-> 經過上面的推導,發現 $p, q$ 皆為偶數,但是 $p, q$ 互質,所以 $p, q$ 不能同時為偶數,矛盾
            \item<7-> 因此 $\sqrt{2}$ 為無理數
        \end{itemize}
    \end{frame}

    \begin{frame}{一些反證法的經典問題}
        \begin{block}{群體中,認識人數問題}
            試證明在一個 $n$ 人的群體中,至少有兩個人認識的人數一樣
        \end{block}

        \begin{block}{算幾不等式的證明}
            已知 $a, b \in \mathbb{R}^+$(正實數),證明 $\frac{a + b}{2} \ge \sqrt{ab}$
        \end{block}
    \end{frame}
    
    \begin{frame}{數學歸納法}
        數學歸納法的步驟還蠻簡單的:
        \begin{enumerate}
            \item 證明 $n = 1$ 時成立
            \item 當 $n = m$ 時,證明 $n = m + 1$ 時成立
        \end{enumerate}

        有點類似我推倒第一張骨牌,接下來每一張骨牌都會因前一張骨牌倒下而倒下
    \end{frame}

    \begin{frame}{數學歸納法}
        \begin{block}{平方和公式}
            試證明 $\sum_{i = 1}^{n}{i ^ 2} = \frac{n(n + 1)(2n + 1)}{6}$
        \end{block}

        \begin{enumerate}
            \item<1-> $n = 1$ 時,成立 
            \item<2-> 當 $n = m$ 時,$\sum_{i = 1}^{m}{i ^ 2} = \frac{m(m + 1)(2m + 1)}{6}$ 成立
            \item<3-> 則 $n = m + 1$ 時,$\sum_{i = 1}^{m}{i ^ 2} + (m + 1)^2 = \frac{m(m + 1)(2m + 1)}{6} + (m + 1)^2$
            \item<4-> $\Rightarrow \sum_{i = 1}^{m}{i ^ 2} + (m + 1)^2 = \frac{m(m + 1)(2m + 1)}{6} + \frac{6(m + 1)(m + 1)}{6}$
            \item<5-> $\Rightarrow = \frac{(m + 1)(m + 2)(2m + 3)}{6}$,滿足 $n = m + 1$ 時的式子
            \item<6-> 由數學歸納法得證
        \end{enumerate}
    \end{frame}

    \begin{frame}{數學歸納法}
        \begin{block}{證明費式數列 $\sum_{i = 0}^{n}{F_i} = F_{n + 2} - 1$}
            $$
            F(x)  = \begin{cases}
                0 \quad & x = 0 \\
                1 \quad & x \le 1 \\
                f(x - 1) + f(x - 2) \quad & x > 1
            \end{cases}
            $$
        \end{block}

        \begin{enumerate}
            \item<1-> $n = 0$ 時,$F_0 = F_2 - 1$,成立
            \item<2-> 假設 $n = m$ 時成立,此時 $\sum_{i = 0}^{m}{F_i} = F_{m + 2} - 1$ 成立
            \item<3-> 在 $n = m + 1$ 時,可得 $\sum_{i = 0}^{m + 1}{F_i} = F_{m + 2} - 1 + F_{m + 1}$
            \item<4-> 可得 $\sum_{i = 0}^{m + 1}{F_i} = F_{m + 3} - 1$
            \item<5-> 由數學歸納法得證
        \end{enumerate}
    \end{frame}

    \begin{frame}{構造性證明}
        \begin{itemize}
            \item 利用建立出一種構造方式,證明假設正確
        \end{itemize}
    \end{frame}

    \begin{frame}{構造性證明}
        \begin{block}{證明質數有無限多個}
            請證明質數有無限多個
        \end{block}

        \begin{itemize}
            \item<1-> 假設已知質數有 $k$ 個 $p_1, p_2, \dots, p_k$
            \item<2-> 那麼我們就可以構造出一個數 $x = \prod_{1}^{k}{p_k} + 1$
            \item<3-> 如果 $x$ 為和數,那麼一定可以找到一個質因數 $P$ 使得 $P \mid x$
            \item<4-> 但這是不可能的,因為 $x \equiv 1 \pmod{p_i}$ 所以,$x$ 為質數,得證
        \end{itemize}
    \end{frame}

    \begin{frame}{回到硬幣問題}
        \begin{block}{硬幣問題}
            你有無限多個面額為 $c_1, c_2, \dots, c_n$ 的硬幣,請問要怎麼用最少的硬幣數量求出 $x$ 元?\\
            已知 $v_1 = 1, v_i \mid v_{i + 1}$($v_i$ 能夠整除 $v_{i + 1}$)
        \end{block}

        \begin{enumerate}
            \item<1-> 假設對於某個 $c_i$ 的硬幣,我們用了超過 $\frac{c_{i + 1}}{c_i}$ 個,
            那麼我們就可以將這 $\frac{c_{i + 1}}{c_i}$ 個 $c_i$ 換成是 $c_{i + 1}$
            \item<2-> 根據以上所述,每個硬幣的數量都不會超過 $\frac{c_{i + 1}}{c_i}$ 個,
            且在最佳解中,若只使用 $c_1 \sim c_{i - 1}$ 的硬幣,所能湊出的最大金額為 $v_{i} - 1$
            \item<3-> 因此當我們需要求出 $x$ 元,且 $c_i \le x < c_{i + 1}$ 時,必選 $c_i$,得證在面額有倍數關係時,貪心解為最佳解
        \end{enumerate}
    \end{frame}

    \section{貪心經典題}

    \begin{frame}{一些很簡單的問題}
        \begin{block}{\href{https://cses.fi/problemset/task/1643}{CSES Maximum Subarray Sum}}
            求最大子陣列和
        \end{block}

        \begin{itemize}
            \item<1-> 還記得區間和怎麼算嗎?
            \item<2-> $\displaystyle\sum_{i = l}^{r}{a_i} = psum_{r} - psum_{l - 1}$ 
            \item<3-> 對於 $r$ 為結尾的最大子陣列和,顯然就是一個最小的 $psum_{i}$ 滿足 $i < r$
            \item<4-> 那麼對於整個陣列的最大子陣列和,答案就是 
            $\displaystyle\max_{r = 1}^{n}(psum_{r} - \max_{i = 1}^{r - 1}(i))$
        \end{itemize}
    \end{frame}

    \begin{frame}{線段貪心}
        \begin{block}{\href{https://cses.fi/problemset/task/1629}{CSES Movie Festival}}
            有 $n$ 場電影,每場電影從 $a_i$ 到 $b_i$,請問最多可以看幾場電影?$(n \le 2 \cdot 10^5)$
        \end{block}

        \begin{itemize}
            \item<1-> 觀察題目,發現可能會有兩種方式:依照開頭排序、依照結尾排序
            \item<2-> 以開頭排序的情況,很簡單的可以發現,若有一部特別早開始的電影,但是特別晚結束,
            那麼這部電影的時間就很有可能包含了其他部電影,導致不會是最佳解
            \item<3-> 那如果以結尾排序呢?假設我們目前在 $x$ 之後有空,那麼根據策略看了一部最早結束的電影 $i$ 之後,
            變為 $b_i$ 之後有空。對於任意一個解所看的電影 $j$ 看完後有空的時間變為 $b_j$,$b_i$ 必定 $\le b_j$
            \item<4-> 顯然在剩下的時間內,$b_i$ 之後所能看的電影數量必定 $\ge b_j$
            \item<5-> 得證不存在另解優於貪心解 $\Rightarrow$ 貪心解為最佳解
        \end{itemize}
    \end{frame}

    \begin{frame}{例題}
        \begin{block}{\href{https://cses.fi/problemset/task/1629}{CSES Movie Festival}}
            有 $n$ 場電影,每場電影從 $a_i$ 到 $b_i$,請問最多可以看幾場電影?$(n \le 2 \cdot 10^5)$
        \end{block}

        \begin{block}{\href{https://cses.fi/problemset/task/1632/}{CSES Movie Festival II(APCS 202301 P4 加強版)}}
            同 Movie Festival,但是共有 $k$ 人能同時看電影,求最多能看幾場電影?(重複不算)
        \end{block}

        \begin{block}{\href{https://tioj.ck.tp.edu.tw/problems/1072}{TIOJ 1072 誰先晚餐}}
            有 $n$ 個人,第 $i$ 個人的餐點需要做 $C_i$ 分鐘、需要吃 $E_i$ 分鐘。廚師一次只能做一道菜,
            但每個人可以同時吃東西,請問最少需要多少時間才能讓所有人吃完?
        \end{block}

        \begin{block}{\href{https://zerojudge.tw/ShowProblem?problemid=b231}{ZJ b231. TOI2009 第三題:書}}
            同 TIOJ 1072
        \end{block}
    \end{frame}

    \begin{frame}{例題}
        \begin{block}{\href{https://cses.fi/problemset/task/1164/}{CSES Room Allocation}}
            有 $n$ 為客人,第 $i$ 位客人的入住時間為 $a_i$,退房時間為 $b_i$,請問最多需要幾間房間來安排客人?並構造出分配方法
        \end{block}
    \end{frame}

    \section{赫夫曼編碼}

    \begin{frame}{赫夫曼編碼}
        \begin{block}{\href{https://cses.fi/problemset/task/1161}{CSES Stick Divisions}}
            你有 $n$ 根棍子 $a_1, a_2, \dots, a_n$,每次操作可以任選兩根棍子 $a_i, a_j$,
            並付出 $a_i + a_j$ 的費用將其合為一根,求最少需要多少費用才能將所有棍子合為一根?
        \end{block}

        \begin{itemize}
            \item<1-> 假設 $n = 1, 2$,那麼答案很顯然
            \item<2-> 那我們來看 $n = 3$ 的情況:合併會經過兩次,而最後一次的費用一定是 $\sum{a_i}$
            \item<3-> 那麼如何讓第一次合併最小便是我們的目標,顯然就是拿兩個最小的合併
            \item<4-> 我們可以猜測,最佳的貪心策略為:每次拿最小的兩個合併
            \item<5-> 如何證明?
        \end{itemize}
    \end{frame}

    \begin{frame}{赫夫曼編碼}
        \begin{block}{\href{https://cses.fi/problemset/task/1161}{CSES Stick Divisions}}
            你有 $n$ 根棍子 $a_1, a_2, \dots, a_n$,每次操作可以任選兩根棍子 $a_i, a_j$,
            並付出 $a_i + a_j$ 的費用將其合為一根,求最少需要多少費用才能將所有棍子合為一根?
        \end{block}

        \begin{itemize}
            \item<1-> 看起來這題很欠數學歸納法
            \item<2-> 顯然 $n \le 2$ 時一定是對的
            \item<3-> 假設 $n = k$ 時策略正確,又 $n = k + 1$ 可由合併任意兩個節點得到 $n = k$ 的情況
            \item<4-> 那麼最佳解即是取兩個最小的元素合併,符合貪心策略
            \item<5-> 由數學歸納法得證,貪心法正確
        \end{itemize}
    \end{frame}

    \begin{frame}{赫夫曼編碼}
        \begin{itemize}
            \item 這個貪心策略便是鼎鼎大名的「赫夫曼編碼 Huffman Coding」
            \item 合併過程產生的赫夫曼樹,又被稱為最優二元樹
            \item 有興趣的可以上網查一下相關資料
        \end{itemize}
    \end{frame}
    
    \section{交換貪心}

    \begin{frame}{交換貪心}
        \begin{itemize}
            \item 其實前面的線段貪心也是交換貪心的一種
            \item 利用先構造出一個解,再嘗試證明出透過某種交換方式可以得到更好的解
            \item 最後再利用得出的交換方式來排序
        \end{itemize}
    \end{frame}

    \begin{frame}{交換貪心}
        \begin{block}{\href{https://cses.fi/problemset/task/1630/}{CSES Tasks and Deadlines}}
            你有 $n$ 個任務,每個任務都有兩個參數 $a, d$,代表完成這個任務的執行時間以及截止時間。\\
            對於每個任務,你所獲得的獎勵為 $d - f$($f$ 代表這個任務最終完成的時間),你每次只能做一個任務,請問最大的獎勵總和是多少?
        \end{block}

        \begin{itemize}
            \item<1-> 首先我們先假設有一個任意解 $S_1$,並拿出任一組相鄰的元素 $i, j\ (j = i + 1)$ 來比較
            \item<2-> 那麼所得的獎勵為:$(d_i - f_i) + (d_j - f_j)$,也就是 $(d_i - f_i) + (d_j - f_i - a_j)$
            \item<3-> 也就是 $(d_i + d_j) - a_j - 2f_i$
            \item<4-> 假設第 $i$ 個任務開始的時間為 $s_i$,那麼式子可以改寫為:$(d_i + d_j) - 2s_i - 2a_i - a_j$
            \item<5-> 我們可以發現到,不管 $i, j$ 的先後順序為何,式子中 $(d_i + d_j) - 2s_i - a_i - a_j$ 都是不變的
            \item<6-> 唯一會變的就只有 $a_i$,因此我們可以得出一個策略:排前面的 $a$ 越小,答案就越大
        \end{itemize}
    \end{frame}

    \begin{frame}{交換貪心}
        \begin{block}{\href{https://cses.fi/problemset/task/1630/}{CSES Tasks and Deadlines}}
            你有 $n$ 個任務,每個任務都有兩個參數 $a, d$,代表完成這個任務的執行時間以及截止時間。\\
            對於每個任務,你所獲得的獎勵為 $d - f$($f$ 代表這個任務最終完成的時間),你每次只能做一個任務,請問最大的獎勵總和是多少?
        \end{block}

        \begin{itemize}
            \item<1-> 那麼該如何證明呢?
            \item<2-> 假設利用貪心法得到的解為 $S$,那麼這個 $S$ 會有一個性質:$a_i \leq a_j,\ i < j$
            \item<3-> 如果這個解並不是最佳解,那麼代表會有一個真正的最佳解 $S'$ 滿足 $a_i > a_j,\ i < j$
            \item<4-> 但根據剛剛的式子推出可以發現,當 $a_i > a_j$ 時,獎勵會變得更少,因此 $S'$ 必定 $\le S$,$S'$ 不滿足最佳解
            \item<5-> 根據反證法證明 $S$ 為最佳解
        \end{itemize}
    \end{frame}

    \begin{frame}{例題}
        \begin{block}{\href{https://cses.fi/problemset/task/1630/}{CSES Tasks and Deadlines}}
            你有 $n$ 個任務,每個任務都有兩個參數 $a, d$,代表完成這個任務的執行時間以及截止時間。\\
            對於每個任務,你所獲得的獎勵為 $d - f$($f$ 代表這個任務最終完成的時間),你每次只能做一個任務,請問最大的獎勵總和是多少?
        \end{block}

        \begin{block}{字串的交換貪心}
            你有 $n$ 個字串,請將這些字串以某種順序接在一起,使得最終的字典序最小
        \end{block}

        \begin{block}{\href{https://codeforces.com/group/H0qY3QmnOW/contest/377732/problem/G}{TPR 20G. 隊伍編排 (Permutation)}}
            請見原題
        \end{block}
    \end{frame}

    \section{數學貪心}
    
    \begin{frame}{數學貪心}
        \begin{block}{\href{https://codeforces.com/problemset/problem/1175/D}{CF 1175D. Array Splitting}}
            有一個數列 $a_1, a_2, \dots, a_n,\ (n \le 3 \cdot 10^5)$,你可以將它分成 $k$ 個子陣列,
            每個元素貢獻的分數為 $a_i \cdot f(i)$($f(i)$ 代表第 $i$ 個元素位處第幾個子陣列)\\
            求最大的分數總和
        \end{block}

        \begin{itemize}
            \item<1-> 假設第 $i$ 個子陣列為 $[l_i, r_i]$,那麼答案可以寫成 $\max\sum_{j = 1}^{k}{suf(l_i)}$($suf(i)$:後綴和,代表 $i \sim n$ 的和)
            \item<2-> 不難發現,最終答案就是從 $n$ 種後綴和中挑出 $k$ 個
            \item<3-> 顯然貪心策略便是前 $k$ 大的後綴和相加
        \end{itemize}
    \end{frame}

    \section{反悔貪心}

    \begin{frame}{反悔貪心}
        \begin{block}{\href{https://codeforces.com/problemset/problem/1526/C2}{CF 1526C2. Potions (Hard Version)}}
            路上總共有 $n$ 瓶藥水,每瓶會對你的血量造成 $a_i$ 點的改變。 $(n \le 2 \cdot 10^5, -10^9 \le a_i \le 10^9)$\\
            你一開始有 $0$ 點血量,你可以選擇每瓶藥水是喝還是不喝,問在血量非負的情況下,最多可以喝幾瓶藥水?
        \end{block}

        \begin{itemize}
            \item<1-> 首先,很顯然的,$a_i \ge 0$ 時一定可以喝
            \item<2-> 但當 $a_i < 0$ 呢?如果喝了還不會死,那當然喝了最好
            \item<3-> 但如果會死呢?
            \item<4-> 程式跟人生不一樣,是可以反悔的。我們可以將之前喝的其他瓶藥水改為不喝,那或許就能喝下這瓶藥水了
            \item<5-> 而選擇哪瓶不喝的策略很簡單,顯然就是扣最多血的不要喝
            \item<6-> 使用 priority queue or set 來維護
        \end{itemize}
    \end{frame}

    \begin{frame}{例題}
        \begin{block}{\href{https://codeforces.com/problemset/problem/1526/C2}{CF 1526C2. Potions (Hard Version)}}
            路上總共有 $n$ 瓶藥水,每瓶會對你的血量造成 $a_i$ 點的改變。 $(n \le 2 \cdot 10^5, -10^9 \le a_i \le 10^9)$\\
            你一開始有 $0$ 點血量,你可以選擇每瓶藥水是喝還是不喝,問在血量非負的情況下,最多可以喝幾瓶藥水?
        \end{block}

        \begin{block}{\href{https://codeforces.com/problemset/problem/1779/C}{CF 1779C. Least Prefix Sum}}
            有一個數列 $a_1, a_2, \dots, a_n$,你可以選擇將 $a$ 中的任意數字 $a_i$ 改為 $-a_i\ (a_i := -a_i)$\\
            請問最少要改幾個數字,才能使得對於所有 $i$,滿足 $\sum_{j = 1}^{i}{a_j} \le \sum_{j = 1}^{m}{a_j}$?$(m \le n \le 2 \cdot 10^5)$
        \end{block}
    \end{frame}

    \section{綜合例題}

    \begin{frame}{綜合例題}
        \begin{block}{\href{https://cses.fi/problemset/task/1631/}{CSES Reading Books}}
            有 $n$ 本書,每本書要花 $t_i$ 的時間讀完。問兩個人最少要花多少時間,才能在不同時看同一本書的情況將所有書看完
        \end{block}

        \begin{itemize}
            \item<1-> 這 $n$ 本書中一定有其中一本書 $i$ 是需要花最久時間的,此時會有兩種情況
            \begin{enumerate}
                \item<2-> $t_i \ge \sum{t} - t_i$ 
                \item<2-> $t_i < \sum{t} - t_i$
            \end{enumerate}
            \item<3-> 第一種情況代表任意一人在讀第 $i$ 本書的期間,另一人可以將其他書全部看完,因此答案 $= 2t_i$
            \item<4-> 至於第二種情況,我們如果將書以 $t_i$ 大至小排序,第一人先看 $t_1$,第二人先看 $t_2$,
            必定不會需要等另外一人讀完某本書,因此時間為 $\sum{t}$,至於為什麼就留給大家自己思考了
            \item<5-> 因此最終的答案便是 $\max(2 \times \displaystyle\max_{i = 1}^{n}t_i, \sum{t})$
        \end{itemize}
    \end{frame}

    \begin{frame}{綜合例題}
        \begin{itemize}
            \item \href{https://codeforces.com/contest/1552/problem/C}{CF 1552C. Maximize the Intersections}
            \item \href{C. Maximize the Intersections}{CF 1506D. Epic Transformation}
            \item \href{https://codeforces.com/problemset/problem/1203/F1}{CF 1203F1. Complete the Projects (easy version)}
        \end{itemize}
    \end{frame}

    \begin{frame}{結語}
        \begin{itemize}
            \item<1-> 貪心看似簡單,但在變化多端的題目上總是埋藏著許多梗
            \item<2-> 貪心題目的關鍵點往往就埋藏在某個性質中
            \item<3-> 看到一道題目後,不妨多多猜測,或許這個擺前面會比較好?或許交換這個會更強?
            \item<4-> 「大膽假設、小心求證」,除了課堂中的題目外,也可以上 CF、CSES 找到相關 tag 的題目多多練習
            \item<5-> 相信大家都能夠學出心得!
        \end{itemize}
    \end{frame}

    \begin{frame}{結語}
        \big|阿我還想再聽貪心怎麼辦?\big|
    \end{frame}

    \begin{frame}{台南一中寒訓!}
        \begin{itemize}
            \item 2/6 $\sim$ 2/11
            \item 全程免費
            \item 各項基礎演算法
            \item \href{http://gg.gg/T25WinterSchedule}{課表}
            \item \href{https://forms.gle/HMuuRkxXUHSjCCjG9}{報名連結(可能會隨時截止)}
        \end{itemize}
    \end{frame}


\end{document}